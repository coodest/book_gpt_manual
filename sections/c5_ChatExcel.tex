\section{ChatExcel: AI辅助制表工具}\label{sec:Excel}

在我们的学习与日常生活中,Excel已然成为一款极为关键的工具,它宛如一位高效的助手,能够显著提升各项事务的处理效率与条理性。
无论是精心规划每日的课程表,精准记录学业成绩,有条不紊地整理学习笔记,还是细致制定学习计划以及合理规划生活预算,Excel都能凭借其清晰明了的结构与简便快捷的操作,轻松应对这些任务。

借助Excel强大的数据分析功能,学生们能够迅速分析成绩走势,挖掘数据背后隐藏的学习状况,或者高效处理调研数据,精准定位学习过程中的薄弱环节。
值得一提的是,Excel还支持多人协作模式,这使得小组作业在分工协作以及进度跟进方面变得轻而易举,极大地提高了团队合作的效率。
熟练掌握Excel,无疑是为优化学习流程开启了便捷之门,更为未来踏入职场以及日常生活的高效管理奠定了坚实的基础。

随着人工智能技术如火箭般迅猛发展,将Excel与GPT技术相融合,为用户带来了前所未有的强大与智能体验。以ChatExcel为典型代表,它创新性地实现了自然语言与电子表格处理之间的无缝交互,彻底颠覆了传统Excel操作模式,让Excel的使用变得更加轻松、高效。
在以往,使用Excel时,用户常常被复杂的公式输入和繁琐的操作步骤所困扰,需要花费大量时间和精力去学习和记忆。而现在,有了ChatExcel,一切都变得简单起来。用户只需以日常对话的形式,清晰地描述自己的需求,强大的GPT便能迅速理解意图,并自动完成数据处理、分析以及可视化呈现等一系列复杂任务。这一变革性的创新,极大地降低了Excel的使用门槛,让更多人能够轻松享受Excel强大功能带来的便利。

这种创新的结合方式,不仅大幅提升了工作效率,让原本需要耗费大量时间和精力的数据处理工作在短时间内就能完成,还打破了专业知识和技能的壁垒,让更多对Excel望而却步的人能够轻松驾驭这一强大工具,充分发挥Excel在数据处理和分析方面的优势。本书以ChatExcel为例。

ChatExcel作为一款融合人工智能(AI)与电子表格处理功能的创新工具,其核心目标就是通过自然语言交互的方式,简化Excel操作流程。它让数据处理不再是专业人士的专利,而是成为大众都能轻松掌握的技能,为我们的学习、工作和生活带来了更多的便利和可能性。
读者可以使用如下步骤安装ChatExcel:

\begin{enumerate}
    \item 安装微软Excel(Office):如果您还没有安装微软Excel,请先下载并安装。ChatExcel需要在Excel中运行。
    \item 打开ChatExcel:在网上搜索ChatExcel,找到ChatExcel官网\footnote{\url{https://chatexcel.com}}。
    \item 使用ChatExcel:点击“现在开始”按钮,就能看到操作页面了。
    \item 上传Excel文件:点击“上传文件”按钮,选择要使用ChatExcel的Excel文件。
    \item 开始使用:在ChatExcel中,您可以使用自然语言与ChatExcel进行交互,并进行数据处理和操作。例如,输入“求和 A1 到 A10”,ChatExcel会自动计算A1到A10的和并返回结果。
\end{enumerate}

下面通过具体的案例讲解ChatExcel的工作过程。

例1 数据清洗:有以下如\reftab{tab:world_ranking1}表格,我们需要将其中的空白格整行删除掉。在ChatExcel上传此Excel文件。
\prompt{删除空白单元格:删除所有有空白单元格的行,处理好的文件,下载给到我}
\begin{table}[h]
    \centering
    \footnotesize
    \begin{tabular}{cccccccc}
        \toprule
        世界排名 & 学校名称 & 地区 & 综合得分 & 学术声誉 & 雇主声誉 & 每位教员引用率 & 师生比 \\
        \midrule
        1  & 麻省理工学院     & 美国 & 100  & 100  & 100  & 100  & 100  \\
        2  & 剑桥大学         & 英国 & 98.8 & 100  & 100  & 92.3 & 100  \\
        3  & 斯坦福大学       & 美国 & 98.5 & 100  & 100  & 99.9 & 100  \\
        4  & 牛津大学         & 英国 & 98.4 & 100  & 100  & 90   & 100  \\
        4  & 某某大学1        & 数据缺失 & 80   & 90   & 数据缺失 & 100  & 数据缺失 \\
        5  & 哈佛大学         & 美国 & 97.6 & 100  & 100  & 100  & 99.4 \\
        6  & 加州理工大学     & 美国 & 97   & 96.5 & 87.1 & 100  & 100  \\
        6  & 帝国理工学院     & 英国 & 97   & 98.3 & 99.7 & 86.5 & 99.5 \\
        8  & 伦敦大学学院     & 英国 & 95   & 99.4 & 98.6 & 77   & 97.6 \\
        9  & 苏黎世联邦理工大学 & 瑞士 & 93.6 & 98.6 & 91.3 & 99.2 & 74.2 \\
        数据缺失  & 未知             & 数据缺失 & 0    & 99   & 0    & 数据缺失 & 数据缺失 \\
        10 & 芝加哥大学       & 美国 & 93.2 & 99.2 & 92.2 & 86.8 & 92.9 \\
        \bottomrule
    \end{tabular}
    \caption{世界大学排名}
    \label{tab:world_ranking1}
\end{table}
\FloatBarrier





\gpt{您好,明白您的需求。接下来的计划是删除表格中存在空白单元格的行,并为您提供处理后的下载文件。
数据处理结果:
根据用户需求,已经成功删除了所有包含空白单元格的行。处理后的表格如\reftab{tab:world_ranking2}:
}
\begin{table}[h]
    \centering
    \footnotesize
    \begin{tabular}{cccccccc}
        \toprule
        世界排名 & 学校名称 & 地区 & 综合得分 & 学术声誉 & 雇主声誉 & 每位教员引用率 & 师生比 \\
        \midrule
        1  & 麻省理工学院     & 美国 & 100  & 100  & 100  & 100  & 100  \\
        2  & 剑桥大学         & 英国 & 98.8 & 100  & 100  & 92.3 & 100  \\
        3  & 斯坦福大学       & 美国 & 98.5 & 100  & 100  & 99.9 & 100  \\
        4  & 牛津大学         & 英国 & 98.4 & 100  & 100  & 90   & 100  \\
        5  & 哈佛大学         & 美国 & 97.6 & 100  & 100  & 100  & 99.4 \\
        6  & 加州理工大学     & 美国 & 97   & 96.5 & 87.1 & 100  & 100  \\
        6  & 帝国理工学院     & 英国 & 97   & 98.3 & 99.7 & 86.5 & 99.5 \\
        8  & 伦敦大学学院     & 英国 & 95   & 99.4 & 98.6 & 77   & 97.6 \\
        9  & 苏黎世联邦理工大学 & 瑞士 & 93.6 & 98.6 & 91.3 & 99.2 & 74.2 \\
        10 & 芝加哥大学       & 美国 & 93.2 & 99.2 & 92.2 & 86.8 & 92.9 \\
        \bottomrule
    \end{tabular}
    \caption{世界大学排名}
    \label{tab:world_ranking2}
\end{table}
\FloatBarrier
同时,我们可以点击预览或者下载删除空格后的文件。我们便可以获得我们想要的修改好的Excel文件。


例2 数据分析-对比分析
首先上传Excel文件,文件中的内容如下表\reftab{tab:1-7}所示:

\begin{table}[h]
    \centering
    \caption{2024年7月份居民消费价格主要数据}
    \footnotesize % 使用\footnotesize使字体变小
    \begin{tabular}{lccc}
        \toprule
        & 环比涨跌幅(\%) & 同比涨跌幅(\%) & 1 - 7月同比涨跌幅(\%) \\
        \midrule
        居民消费价格 & 0.5 & 0.5 & 0.2 \\[-3pt]
        其中:城市 & 0.6 & 0.5 & 0.2 \\[-3pt]
        农村 & 0.4 & 0.7 & 0.2 \\[-3pt]
        其中:食品 & 1.2 & 0.0 & -2.3 \\[-3pt]
        非食品 & 0.4 & 0.7 & 0.8 \\[-3pt]
        其中:消费品 & 0.4 & 0.5 & -0.3 \\[-3pt]
        服务 & 0.6 & 0.6 & 0.9 \\[-3pt]
        其中:不包括食品和能源 & 0.3 & 0.4 & 0.6 \\[-3pt]
        \midrule
        按类别分 & & & \\[-3pt]
        一、食品烟酒 & 0.7 & 0.2 & -1.2 \\[-3pt]
        粮食 & -0.3 & 0.1 & 0.4 \\[-3pt]
        食用油 & -0.6 & -4.1 & -4.9 \\[-3pt]
        鲜菜 & 9.3 & 3.3 & -1.9 \\[-3pt]
        畜肉类 & 0.8 & 4.9 & -2.5 \\[-3pt]
        其中:猪肉 & 2.0 & 20.4 & 2.7 \\[-3pt]
        牛肉 & -0.9 & -12.9 & -10.3 \\[-3pt]
        羊肉 & -0.5 & -6.3 & -6.2 \\[-3pt]
        水产品 & 0.4 & 1.2 & 0.7 \\[-3pt]
        蛋类 & 3.7 & 0.4 & -6.0 \\[-3pt]
        奶类 & -0.1 & -1.9 & -1.5 \\[-3pt]
        鲜果 & -0.3 & -4.2 & -7.3 \\[-3pt]
        卷烟 & 0.1 & 1.0 & 1.2 \\[-3pt]
        酒类 & 0.2 & -1.6 & -1.2 \\[-3pt]
        \midrule
        二、衣着 & -0.4 & 1.5 & 1.6 \\[-3pt]
        服装 & -0.3 & 1.7 & 1.7 \\[-3pt]
        鞋类 & -0.5 & 0.3 & 0.8 \\[-3pt]
        \midrule
        三、居住 & 0.0 & 0.1 & 0.2 \\[-3pt]
        租赁房房租 & 0.1 & -0.3 & -0.1 \\[-3pt]
        水电燃料 & 0.0 & 0.9 & 0.6 \\[-3pt]
        \midrule
        四、生活用品及服务 & 0.4 & 0.7 & 0.9 \\[-3pt]
        家用器具 & 0.4 & -1.8 & -0.8 \\[-3pt]
        家庭服务 & 0.1 & 1.7 & 1.8 \\[-3pt]
        \midrule
        五、交通通信 & 0.9 & -0.6 & -0.7 \\[-3pt]
        交通工具 & -0.6 & -5.6 & -5.1 \\[-3pt]
        交通工具用燃料 & 1.5 & 5.1 & 3.8 \\[-3pt]
        交通工具使用和维修 & 0.1 & 0.5 & 0.5 \\[-3pt]
        通信工具 & -0.1 & -2.1 & -2.2 \\[-3pt]
        通信服务 & 0.0 & 0.0 & -0.3 \\[-3pt]
        邮递服务 & -0.1 & -0.5 & -0.4 \\[-3pt]
        \midrule
        六、教育文化娱乐 & 1.3 & 1.7 & 2.0 \\[-3pt]
        教育服务 & 0.0 & 1.8 & 1.7 \\[-3pt]
        旅游 & 9.4 & 3.1 & 6.4 \\[-3pt]
        \midrule
        七、医疗保健 & 0.0 & 1.4 & 1.4 \\[-3pt]
        中药 & 0.0 & 2.9 & 4.9 \\[-3pt]
        西药 & -0.2 & -1.0 & -0.7 \\[-3pt]
        医疗服务 & 0.0 & 1.9 & 1.7 \\[-3pt]
        \midrule
        八、其他用品及服务 & 0.9 & 4.0 & 3.4 \\[-3pt]
        \bottomrule
    \end{tabular}
    \label{tab:1-7}
\end{table}
\FloatBarrier

上传完成后,发送如下命令:

\prompt{对比分析7月和8月的居民消费价格,图文并茂的}

ChatExcel给出了如下内容:
\gpt{
    \begin{enumerate}
        \item \noindent\textbf{数据提取与对比}
        
            根据提供的数据,我们成功提取并对比了7月和8月的居民消费价格及其分类(城市、农村、食品、非食品)的各项涨跌幅情况。具体内容如下:
            \begin{itemize}
                \item \noindent\textbf{数据概述}
                
                    8月数据:
                    \begin{itemize}
                        \item 居民消费价格:环比涨跌幅 \(0.4\%\),同比涨跌幅 \(0.6\%\) 
                        \item 城市:环比涨跌幅 \(0.3\%\),同比涨跌幅 \(0.6\%\) 
                        \item 农村:环比涨跌幅 \(0.4\%\),同比涨跌幅 \(0.8\%\) 
                        \item 食品:环比涨跌幅 \(3.4\%\),同比涨跌幅 \(2.8\%\) 
                        \item 非食品:环比涨跌幅 \(-0.3\%\),同比涨跌幅 \(0.2\%\) 
                    \end{itemize}
                    7月数据:
                    \begin{itemize}
                        \item 居民消费价格:环比涨跌幅 \(0.5\%\),同比涨跌幅 \(0.5\%\) 
                        \item 城市:环比涨跌幅 \(0.6\%\),同比涨跌幅 \(0.5\%\) 
                        \item 农村:环比涨跌幅 \(0.4\%\),同比涨跌幅 \(0.7\%\) 
                        \item 食品:环比涨跌幅 \(1.2\%\),同比涨跌幅 \(0\%\) 
                        \item 非食品:环比涨跌幅 \(0.4\%\),同比涨跌幅 \(0.7\%\) 
                    \end{itemize}
            \end{itemize}
        \item \noindent\textbf{各项差异分析}
        
            通过计算各项指标在7月和8月之间的差异,我们得出以下结论:
            % 此处仅保留对表格的描述,实际表格移到\gpt外面
            各项指标在7月和8月之间的差异统计如表\reftab{tab:price_change_difference_1}所示,另外,我们还统计了包含1 - 8月同比涨跌幅差异的数据,具体内容见表格\reftab{tab:price_change_difference_2}。
        \item \noindent\textbf{图表展示}
        
            为了更直观地展示7月和8月的数据变化,我们绘制了对比图表。这些图表可以帮助更好地理解各项指标的变化趋势。
        \item \noindent\textbf{结论}
        
            总体来看,8月的居民消费价格环比涨幅有所回落,而同比涨幅略有上升。其中,食品价格在8月出现较大涨幅,环比和同比分别上涨了 \(3.4\%\) 和 \(2.8\%\),较7月有显著提升。相反,非食品价格在8月出现了环比下降,降幅为 \(0.3\%\)。城市和农村的消费价格变化相对平稳,但农村的同比涨幅略高于城市。
            
            通过对7月和8月数据的详细对比分析,可以发现食品价格是影响整体居民消费价格的主要因素之一,而非食品价格则表现较为稳定。结合新增的1 - 8月同比涨跌幅差异数据,能更全面地了解价格变化趋势。
    \end{enumerate}
}


\begin{table}[h]
    \centering
    \footnotesize
    \begin{tabular}{lcc}
        \toprule
        类别 & 环比涨跌幅差异 & 同比涨跌幅差异 \\
        \midrule
        居民消费价格 & \(0.4\% - 0.5\%=-0.1\%\) & \(0.6\% - 0.5\% = 0.1\%\) \\
        城市 & \(0.3\% - 0.6\%=-0.3\%\) & \(0.6\% - 0.5\% = 0.1\%\) \\
        农村 & \(0.4\% - 0.4\% = 0\%\) & \(0.8\% - 0.7\% = 0.1\%\) \\
        食品 & \(3.4\% - 1.2\% = 2.2\%\) & \(2.8\% - 0\% = 2.8\%\) \\
        非食品 & \(-0.3\% - 0.4\%=-0.7\%\) & \(0.2\% - 0.7\%=-0.5\%\) \\
        \bottomrule
    \end{tabular}
    \caption{7月和8月各项指标涨跌幅差异}
    \label{tab:price_change_difference_1}
\end{table}
\FloatBarrier

\begin{table}[h]
    \centering
    \footnotesize
    \begin{tabular}{lccc}
        \toprule
        指标名称 & 环比涨跌幅差异 & 同比涨跌幅差异 & 1 - 8月同比涨跌幅差异\\
        \midrule
        居民消费价格 & -0.1\% & 0.1\% & 0.0\%\\
        城市 & -0.3\% & 0.1\% & 0.0\%\\
        农村 & 0.0\% & 0.1\% & 0.1\%\\
        食品 & 2.2\% & 2.8\% & 0.6\%\\
        非食品 & -0.7\% & -0.5\% & -0.1\%\\
        \bottomrule
    \end{tabular}
    \caption{指标差异数据}
    \label{tab:price_change_difference_2}
\end{table}
\FloatBarrier