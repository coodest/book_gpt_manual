\hypertarget{AIux5efaux6a21}{\section{AI建模}} \subsubsection{案例:基础3D建模}\label{ux6848ux4f8b2ux5386ux53f2ux8fdbux7a0bux6d41ux7a0bux56fe}

\textbf{背景}:通过AI工具快速实现3D场景原型搭建

\textbf{实现方法}: \begin{enumerate} \def\labelenumi{\arabic{enumi}.} \item \textbf{提示词设计} \prompt{写Blender Python脚本:创建一个简单立方体,包含基础灯光和摄像机}

\item \textbf{GPT生成核心代码} 
\begin{gptbox} 
\small{ 
\begin{lstlisting}
import bpy

# 清空场景
bpy.ops.object.select_all(action='SELECT')
bpy.ops.object.delete(use_global=False)

# 创建立方体
bpy.ops.mesh.primitive_cube_add(
    size=2,                # 立方体边长
    location=(0, 0, 0),    # 中心位置
    rotation=(0, 0, 0)     # 初始旋转
)
cube = bpy.context.object
cube.name = "Simple_Cube"  # 命名对象

# 创建基础材质
material = bpy.data.materials.new(name="Cube_Material")
material.diffuse_color = (0.8, 0.1, 0.1, 1)  # RGBA红色
cube.data.materials.append(material)

# 设置摄像机
bpy.ops.object.camera_add(
    location=(5, -5, 3),   # 摄像机位置
    rotation=(1.047, 0, 0.785)  # 60度俯角,45度偏角
)
bpy.context.scene.camera = bpy.context.object

# 添加灯光
bpy.ops.object.light_add(
    type='POINT',
    location=(3, -3, 5),
    radius=1.0
)
bpy.context.object.data.energy = 500  # 灯光强度

# 设置渲染基础参数
bpy.context.scene.render.resolution_x = 1280
bpy.context.scene.render.resolution_y = 720

print("基础立方体场景创建完成!")

\end{lstlisting}
} 
\end{gptbox}

\item \textbf{教学延伸} \prompt{如何修改立方体尺寸和颜色?}

GPT回复: 
\begin{gptbox} 
\small{ 
\begin{itemize} 
\item 
修改尺寸:调整
\verb|size=|参数值 
\item 修改颜色:改变
\verb|(0.8, 0.1, 0.1, 1)|的RGB数值 
\end{itemize} } 
\end{gptbox} 
\end{enumerate}

\hypertarget{ux5b9eux8df5ux6a21ux5757}{\subsection{动手试试看}} 
AI提示词示例 
改变物体形状:
\prompt{将立方体改为球体}\\
调整场景光照:
\prompt{添加三点照明系统} \\
制作简单动画:
\prompt{创建旋转动画效果} \\