\hypertarget{ux6848ux4f8b3ux5c06ux62cdux6444ux7684ux7167ux7247ux4feeux6539ux4e3aux52a8ux6f2bux98ceux683c}{%


\hypertarget{ux6848ux4f8b43dux6587ux7269ux91cdux5efa}{%
\subsubsection{案例4:3D文物重建}\label{ux6848ux4f8b43dux6587ux7269ux91cdux5efa}}

\textbf{背景}:数字化复原破损的唐三彩马俑

\textbf{实现方法}:

\begin{enumerate}
\def\labelenumi{\arabic{enumi}.}
\item
  \textbf{提示词设计} 
  \prompt{``写Blender Python脚本:创建唐代马俑基础模型,特征包含:''

  \begin{itemize}
  
  \item
    颈部弯曲呈S型
  \item
    马鞍有莲花纹浮雕
  \item
    表面材质为陶器质感
  \end{itemize}
  }
\item
  \textbf{GPT生成核心代码}

\begin{lstlisting}
# 创建马体基础形态
bpy.ops.mesh.primitive_cylinder_add(vertices=64)
bpy.ops.object.modifier_add(type='SUBSURF')
# 添加纹理
material = bpy.data.materials.new(name="Tang_Sancai")
material.use_nodes = True
bsdf = material.node_tree.nodes["Principled BSDF"]
bsdf.inputs['Base Color'].default_value = (0.8,0.3,0.1,1) # 橙红色陶釉
\end{lstlisting}
\item
  \textbf{细节优化} ``如何用几何节点创建破损效果?'' GPT回复:

\begin{lstlisting}
# 添加几何节点修改器
node_group = bpy.data.node_groups.new("Crack_Effect", 'GeometryNodeTree')
# 设置噪波纹理控制顶点位移...
\end{lstlisting}
\end{enumerate}



\hypertarget{ux4f26ux7406ux4e0eux6ce8ux610fux4e8bux9879}{%
\subsection{伦理与注意事项}\label{ux4f26ux7406ux4e0eux6ce8ux610fux4e8bux9879}}

\begin{enumerate}
\def\labelenumi{\arabic{enumi}.}

\item
  \textbf{版权声明}:AI生成内容需标注来源
\item
  \textbf{事实核查}:历史细节需对照权威资料
\item
  \textbf{创意平衡}:避免过度依赖AI导致同质化
\end{enumerate}



\hypertarget{ux53efux89c6ux5316ux5b66ux4e60ux5de5ux5177ux5305}{%
\subsection{可视化学习工具包}\label{ux53efux89c6ux5316ux5b66ux4e60ux5de5ux5177ux5305}}

\begin{itemize}

\item
  Blender GPT插件:AI Assistant for Blender(直接对话生成脚本)
\item
  流程图协作平台:Diagrams.net + GPT集成
\item
  历史模型库:Sketchfab开源3D模型数据库
\end{itemize}



这样的结构既保证了专业深度,又通过真实可操作的案例(包含可直接复制使用的代码段)降低学习门槛。建议在每个案例后添加``动手实验室''板块,提供分步骤练习任务,例如:``尝试用GPT生成北宋汴京城的市集场景脚本''。