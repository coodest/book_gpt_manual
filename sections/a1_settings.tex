% margin of the page
\fancypagestyle{main}{
    \fancyhf{}
    \renewcommand\headrulewidth{1pt}  % 页眉横线
    \renewcommand\footrulewidth{0pt}  % 页脚横线
    
    \fancyhead[OR]{\fzkai{\leftmark}}  % 页眉章标题
    \fancyhead[EL]{\fzkai{\rightmark}}  % 页眉节标题
    \fancyfoot[OC,EC]{\thepage}  % 页码
}
\pagestyle{main}
\geometry{
  a4paper,
  left=3cm,
  right=3cm,
  top=2.54cm,
  bottom=2.54cm,
  headheight=1.5cm,
  headsep=10pt
}


% links
\hypersetup{
    CJKbookmarks=true,
    colorlinks=true,
    breaklinks=true,
}


% graphics
\graphicspath{ {assets/figures/} }
\DeclareGraphicsExtensions{.png, .pdf}


% theorems
\theoremstyle{plain}
\newtheorem{theorem}{Theorem}[section]
\newtheorem{proposition}[theorem]{Proposition}
\newtheorem{lemma}[theorem]{Lemma}
\newtheorem{corollary}[theorem]{Corollary}
\theoremstyle{definition}
\newtheorem{definition}[theorem]{Definition}
\newtheorem{assumption}[theorem]{Assumption}
\theoremstyle{remark}
\newtheorem{remark}[theorem]{Remark}


% colors
\definecolor{g1}{HTML}{7be495}
\definecolor{b1}{HTML}{1eb0bb}


% chatting box
\tcbuselibrary{breakable, listings} %breakable:支持跨页
\newtcolorbox[
    auto counter, 
    number within=section, 
    list type=section, 
    list inside=toc]{promptbox}[1]{
        colback=white, 
        colframe=g1, %背景和线条颜色
        colbacktitle=g1, 
        coltitle=white, %标题框的背景和线条颜色 
        fonttitle=\bfseries, 
        title={读者提问 \thetcbcounter}, 
        list entry={读者提问 \thetcbcounter\quad}, %标题
        breakable, %支持跨页
        before upper={\parindent10pt\noindent},  % 支持缩进。\noindent:首行不缩进
        left = 1 mm, %文字离线框左边的边距
        right = 1 mm, %同上
        top = 1 mm, %同上
        bottom = 1 mm,%同上
        left skip = 0.25\textwidth,
        % arc is angular = 1mm, % 棱角线框
        % sharp corners, % 直角线框
        % enhanced,frame hidden, % 隐藏线框
        enhanced,drop fuzzy shadow,  % 显示阴影
}

\newtcolorbox[
    auto counter, 
    number within=section, 
    list type=section, 
    list inside=toc]{gptbox}[1]{
        colback=white, 
        colframe=b1, %背景和线条颜色
        colbacktitle=b1, 
        coltitle=white, %标题框的背景和线条颜色 
        fonttitle=\bfseries, 
        title={AI作答 \thetcbcounter}, 
        list entry={AI作答 \thetcbcounter\quad}, %标题
        breakable, %支持跨页
        before upper={\parindent10pt\noindent},  % 支持缩进。\noindent:首行不缩进
        left = 1 mm, %文字离线框左边的边距
        right = 1 mm, %同上
        top = 1 mm, %同上
        bottom = 1 mm,%同上
        right skip = 0.10\textwidth,
        % arc is angular = 1mm, % 棱角线框
        % sharp corners, % 直角线框
        % enhanced,frame hidden, % 隐藏线框
        enhanced,drop fuzzy shadow,  % 显示阴影
}

\lstset{
    basicstyle=\ttfamily\small,  % 字体和大小
    commentstyle=\color{gray},  % 注释颜色
    keywordstyle=\color{blue},  % 关键字颜色
    stringstyle=\color{red},    % 字符串颜色
    breaklines=true,            % 自动换行
    numbers=left,               % 行号在左侧显示
    numbersep=-10pt,
    % frame=shadowbox,            % 添加阴影边框
    % rulesepcolor=\color{gray}   % 边框颜色
}

