\section{GPT辅助搜索}\label{sec:gpt-aided-writing}
\subsection{Paper Search Engine(文献搜索引擎)}

要最新的学术论文和摘要时,这个搜索引擎能够快速地为你提供帮助。它擅长搜索并总结最新的学术成果,非常适合进行前期的文献回顾和信息收集。

\textbf{它可以做什么,示例用法:}
\begin{itemize}
    \item 查询最近关于具化智能体(Embodied Agent)的研究论文
    \item 总结一下最新的具化智能体研究进展
    \item 对具化智能体领域的研究进行一个简要的概览
    \item 对所提供的具化智能体相关论文进行摘要总结
\end{itemize}
\textbf{地址:} \url{https://chat.openai.com/g/g-9v5gHG9Bo}

\subsection{Academic Paper Specialist(学术论文撰写专家)}
\textbf{它可以做什么,示例用法:}
\begin{itemize}
    \item 优化这篇论文摘要
    \item 检查这篇论文的行文逻辑
    \item 使这段论文听起来更符合中文语境的表达
    \item 评估这一段落的逻辑性
\end{itemize}
\textbf{地址:} \url{https://chat.openai.com/g/g-jryw3pfsH}

\subsection{Paper Connect 论文整理助手}
\textbf{它可以做什么,示例用法:}
\begin{itemize}
    \item 请帮我找最近的文章在Nature子刊上
    \item 我想知道近期在IEEE Transactions on Pattern Analysis and Machine Intelligence上发表的最新研究
    \item 找一下近期在Journal of Neuroscience上的最新文章
    \item 查询最近在MICCAI会议上发表的论文
\end{itemize}
\textbf{地址:} \url{https://chat.openai.com/g/g-Fv62pCdo2}

\subsection{QUICK WRITING ACADEMIC'S PAPERS(论文快速撰写工具)}
这个工具可以帮助你从论文的“主题”开始,逐步深入。通过分节进行讨论,它能够辅助你一步步构建论文的各个部分。
\textbf{地址:} \url{https://chat.openai.com/g/g-iTycjneA8}

\subsection{Academic Paper Creator(学术论文创作助手)}
对于需要使用 LaTeX 写作或者需要特定PDF格式设置的论文来说,这个GPTs可以提供相当大的辅助。它可以帮助你更加高效地完成论文的撰写和格式设置。
\textbf{地址:} \url{https://chat.openai.com/g/g-DzTFVQytf}

\subsection{论文润色大师}
\textbf{作用:} 优化你的学术论文,润色语言
\textbf{它可以做什么,示例用法:}
\begin{itemize}
    \item 编辑这句话以提高论文表述的清晰度
    \item 这个段落的流畅性该如何提高
    \item 为这篇学术文本提出改进建议
    \item 将这个长句拆分以提升可读性
\end{itemize}
\textbf{地址:} \url{https://chat.openai.com/g/g-UPuGbvUJn}

\subsection{Paper Reviewer(论文评审助手)}
\textbf{作用:} 它可以充当一个细心的同行评审员,帮你评审论文。不管是概述数据科学研究,比如列出机器学习论文的亮点,还是指出计算机科学研究的不足之处,这个工具都能派上用场。
\textbf{它可以做什么,示例用法:}
\begin{itemize}
    \item 优化这篇论文摘要
    \item 检查这篇论文的行文逻辑
    \item 使这段论文听起来更符合中文语境的表达
    \item 评估这一段落的逻辑性
\end{itemize}
\textbf{地址:} \url{https://chat.openai.com/g/g-8Sfcmardr}

\subsection{Paper Reframer}
这个工具能够帮助你改写学术论文。无论是重述摘要,简化文本,还是准确地改写某一部分内容,它都能提供有效的帮助,名副其实的文献综述神器。
\textbf{它可以做什么,示例用法:}
\begin{itemize}
    \item 请重新表述这个摘要,使其更清晰
    \item 请简化这一段落
    \item 准确地改述这一部分内容
    \item 帮助重写这篇论文的摘要
\end{itemize}
\textbf{地址:} \url{https://chat.openai.com/g/g-pyqYuYOjE}

\subsection{AI fact-checking paper(AI事实核查论文解读)}
在处理涉及事实核查的论文时,这个工具能够帮助你理解并解读相关的研究。它能解释论文的核心观点,总结关键发现,甚至批判 AI 在事实核查中的角色。
\textbf{它可以做什么,示例用法:}
\begin{itemize}
    \item 你能解释一下这篇论文的主要论点吗
    \item 关于人工智能在事实核查中的关键发现是什么
    \item 这篇论文是如何批评人工智能在事实核查中的作用的
    \item 你能总结一下这项研究的结论吗
\end{itemize}
\textbf{地址:} \url{https://chat.openai.com/g/g-fuEoRgOsG}
