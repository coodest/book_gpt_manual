\section{介绍}\label{sec:intro}
GPT(Generative Pre-trained Transformer)是一种基于Transformer架构的大型语言模型,主要用于自然语言处理和生成任务。最初由OpenAI开发,如今全球许多公司和研究机构也纷纷涉足类似技术的开发,例如国内的百度ERNIE、阿里巴巴“飞扬写作”、腾讯PLATO,以及国外的LinkedIn、Microsoft等。这些模型在学术、商业、教育等多个领域得到了广泛应用,包括文献搜索、论文阅读、写作辅助、智能客服、内容创作、个性化学习、语言教学等,显著提高了工作效率和质量。

在论文写作中,GPT及其类似技术的应用尤为突出。其功能涵盖了从论文结构设计到语言润色的全方位支持,能够快速生成论文各部分(如引言、方法、结果、讨论)的内容。例如,通过自然语言描述输入查询需求,可以快速获取相关文献,甚至使用特定工具(如“Paper Search Engine”)总结最新的学术成果。此外,GPT还能对论文进行详细解读,快速总结关键内容,提取核心观点,解释复杂术语。然而,需要注意的是,GPT生成的内容需要经过仔细校对和验证,以确保其准确性和可靠性。